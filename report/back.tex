\section{Background}
\label{sec:background}

Automating experiment design is not without precedent.  Kristine Smith started the field of optimal experiment design in 1918 \cite{smith1918standard}.
It is only recently that robots have been employed to conduct scientific exploration autonomously \cite{wagner2001science},\cite{king2004functional}.  We will discuss below is how current robot scientists reliance on global information prevents them from operating in truly unknown environments.  Additionally, we will cover how previous approaches in sequential decision making from statistics do not necessarily reflect the settings that autonomous robots encounter in the real world.


% 	Who else has done what?
% 
% - Evidence that humans make (approximately) rational decisions (we over and under estimate low and high probabilities)
% 
% 	- Design of experiments has led to (amongst other things) multi-armed bandit models of sequential experiment design.
% 	- See also maximum entropy sampling
% 	- See also mutual information sampling.
% - Also consider active learning solutions (they all end up being the same anyway)
% 
% 	- Robotics research has made robots that conduct exploration, but the only ones that make decisions about whether to investigate something or not do one of three things:
% 	1. match templates.
% 	2. seek improbable things.
% 	3. Engage in opportunistic science - they do something if they have the time.  They don't override human mission objectives.
% 
% 	1 and 2 say nothing about the information content of the material under investigation.
% 	3 does not have the level of autonomy that we need for truly long-term or remote operations.
% 
% 	How?
% 
% 	What have we previously done:
% 	- D.R. Thompson's work
% 	- Only looking at satellite imagery.  Good but not sufficient.
% 	- Trey's work
% 	- Using POMDPs not scalable to a planet.
% 	- Mine. Where does my previous work fall short?
% - Not bayesian (not a big deal?)
% 	- Still has the problem on knowing the number of sampling opportunities remaining.

\subsection{The Secretary Problem}

The secretary problem\footnote{Also known as the fiance problem or the game of
Googol} asks a decision maker to select the best candidate from sequentially
presented candidates where it is not possible to return to previously rejected
candidates.  In the original setting there is only one position to for the
candidate to fill \cite{ferguson1989solved}, and the optimal strategy is to
reject the first $\frac{N}{e}$ candidates and then take the first candidate who
is ranked better than any of the previously seen candidates or hte final
candidate.  In the original setting the decision maker was able to objectively
rank the candidates.   

There have been many variations on this problem including selecting multiple
candidates \cite{vanderbei1980optimal}, or when the total number of candidates
is random \cite{presman1972best}, for more variations \cite{ferguson1989solved}
is an excellent source.  What distinguishes the secretary problem from the
science autonomy problem we propose is that we do not know the value of a
candidate, or class, when we encounter it.  Additionally repeatedly sampling
the same class decreases the value to the decision maker. 


% 		- Secretary problem
% 			- They know what the value of their choice is, in many cases.
% 			- They know how many candidates there are.
% 			- We have rates of arrival, they just have a queue.
% 			- While our sampling time cost implies an upper bound on the number of 
% 			candidates we don't have a lower bound necessarily.
% 
% 			\cite{ferguson1989solved}, \cite{presman1972best}, \cite{vanderbei1980optimal}

\subsection{Multi-armed bandits}

Sequential experiment selection, a type of active learning, is addressed in the
multi-armed bandit literature.  The multi-armed bandit was introduced in
\cite{robbins1952some} as a means of sequentially selecting which experiments
to conduct with a limited budget.  In Robbins' work \cite{robbins1952some}
selecting experiments is modelled on determining the payouts of one-armed
bandit machines -- each machine represents a different experiment.  The player
has a fixed sampling budget and has to sequentially choose which machine to
play, trading off exploiting the expected rewards for the different arms and
exploring the different arms learning more accurately the payouts of those
arms.  

Lai \emph{et al.} \cite{lai1985asymptotically} introduced the Upper Confidence
Bound (UCB) rule which values sampling opportunities with the sum of the
expected reward for a sampling opportunity and a term that tries to balance the
samples amongst all types of sampling opportunities.

$$
Value = \mathbb{E}\left[R_i\right] + \sqrt{\frac{2\ln t_i}{T}}
$$

Where $R_i$ is the reward for sampling opportunity $i$, $t_i$ is the number of
times $i$ has been sampled, and $T$ is the total number of samples distributed.
Work on proving the bounds of this algorithm has been continued by Agarawal
\cite{agrawal1995sample} and Auer and Ortner\cite{auer2010ucb}.  


Other approaches to the bandit problem use reward plus the uncertainty of that
reward to indicate value.  We see this in the work of Burnetas and Katehakis
\cite{burnetas1997optimal} and Auer \cite{auer2003using}.  This is a sentiment
seen in other work, like the optimistic planners of Jurgen Schmidhuber's group
\cite{schmidhuber1997what,schmidhuber2003exploring,schmidhuber2009simple,sun2011planning}.
They choose actions that maximize the expected information gain with respect to
some model they are learning.  The most valuable actions are the ones that
result in the greatest shift in the distribution the learner is building.

Balcan \cite{balcan2006agnostic} presents a method for learning classifiers by
requesting samples from the input space with the greatest classification
error.  Classification error and uncertainty in function value
are fungible quantities in this case.  An analogy can be drawn between
the classifiers used in \cite{balcan2006agnostic} and the bandit arms used by
Auer and Ortner\cite{auer2010ucb}.

% 		- Multi-armed bandit
% 			- Assumes you can access any arm at any time
% 			- Many settings don't have a switching cost between arms.
% 			- Gittins showed that if you have a switching cost and diminishing returns you can solve the problem.
% 			- We say is we will be randomly assigned an arm and a switching cost.

There are a number of distinguishing factors between the multi-armed bandit
(MAB) setting and the setting explored in this paper.  First, in MAB the agent
has access to any arm it chooses at any given time.  The arms in MAB are
analogous to the classes in our setting.  The agent in our setting does not get
to choose which of the classes it can investigate.  Any previously seen classes
are no longer available and new classes are randomly assigned.  Additionally
the standard MAB setting does not have switching costs, although there are some
formulations which do account for switching costs
\cite{switchingBanditPapersFindOne}.  In our setting there is a cost incurred
with every choice to continue exploring, and it is a function of the arrival
rates of the different classes.

\subsection{Optimal Foraging}

Foraging is the problem encountered by animals seeking to maximize the intake
of energy when operating in an unknown environment.  The central question to
solving the problem is: Is it more valuable to continue extracting resources
from the current location or to seek out new locations to extract resources
from?  Charnov introduced a technique for dealing with what he called
``patchy'' environments, where there are localized regions that contain
different classes of resources \cite{charnov1973optimal},
\cite{charnov1976optimal}.  The forager can extract value from these patches,
with diminishing returns, or choose to continue to wander randomly through the
environment in the hopes of encountering a more valuable location.

The time to leave the environment, according to the Marignal Value Theorem
\cite{charnov1973optimal}, was when the expected return from continuing to
sample from this patch is less than the expected return from wandering in the
environment.  In this formulation the expected return from both the current
patch and the environment are offset by the cost of extracting resources in
this patch as well as the energy spent seeking a new patch to extract resources
from.

% 	- Work in the 1970's about foraging.  About making value judgements.  - Key
% 	point from Charnov's work is that there has to be diminishing returns for
% 	extracting from a field (specifically towards an asymptote) - Different from
% 	our setting is that diminishing your reward in one area contributes to
% 	diminishing your reward elsewhere. It isn't like picking apples off one tree,
% 	and finding a new tree with unpicked apples.
% 	\cite{charnov1973optimal},\cite{charnov1976optimal}.
 


\cite{pirolli1999information} introduced a model of researchers attempting to
acquire information.  They modelled the rate of information gain and had their
agent decide to leave the patch they were in when the rate of information gain
was lower than that of the environment.  What differentiates their setting from
ours is that their decision maker can choose from which reservoirs to sample.  Our exploring agent does not have that luxury.

% 	- Piroli 1999 - has come up with a very similar formulation as the one that 
% 		Mike and I came up with.
% 		- Something to consider is that we only get to take one sample per patch 
% 			because of science differentiation requirements (i.e. RP can't have 
% 			samples closer than 10cm to each other, maybe as far as 1m)
% 		- We could also incorporate the different cost it takes to identify things 
% 			in the scene.
% 		- We also have different times between patches.
% 		- Should it come to multiple contexts then we have the contextual bandit 
% 			problem
% 				- This means that we could keep different distributions of classes per 
% 					environment.

\cite{kolling2012neural} studies how humans engage in a gambling task where
they have to consider the option they have before them and the opportunities
the environment provides.  In the described experiment subjects were repeatedly
presented with a choice of playing a gambling game or being randomly presented
with a different game.  Each game was a Bernoulli trial with some unknown
probability of success.  Kolling \emph{et al.} identify possible neural
substrates for foraging decision making in humans.  The behaviour was near
optimal, with some skewing of probabilities at the extreme ends of the scale,
i.e. $p \approx 0$ or $p \approx 1$.


\subsection{Science Autonomy}

% 	- Yeorger (sp?) WHOI or MBARI has exactly this setup as a problem.
% 	- Thompson, Asher Bender and Stephane Williams Group
% 			- making selections based on maps.  
% 			- Maximum entropy/mutual information smapling.
% 			- They assume global knowledge, we don't have that.  It is reasonable
% 			in many settings of interest, can reduce cost.

Thompson and Wettergreen \cite{thompson2008intelligent} maximize diversity of
collected samples by using mutual information sampling.  This approach ensures
diversity in the collected sample set, an act that reduces uncertainty in the
input space of a function.  Neither mutual information nor maximum entropy
sampling methods, when used with stationary Gaussian processes, take into
account the dependent variable when selecting samples.  


\cite{bender2013autonomous} makes a modification to the work of
\cite{thompson2008intelligent}, in that it uses Gaussian processes to
identified hypothesized distributions of life across the sea floor to direct
exploratory actions.  The prior maps were generated by vessels passing over the
sea floor prior to the robot's exploration mission, not unlike using satellite
imagery in \cite{thompson2008intelligent}.  The advance of
\cite{bender2013autonomous} is that it uses \emph{in situ} measurements to
update the Gaussian process it is learning.  Bender \emph{et al.}'s rover can
be said to be generating and testing hypotheses.  However they are severely
limited by a budgeting size of six ``gulpers'' -- devices for collecting
seawater samples.


Ferri \emph{et al.} \cite{ferri2010novel} present an approach to prospecting where an AUV follows a predefined track and needs to decide when deviate to sample anomalies.  The AUV in this work examines anomalies by engaging in a spiral search pattern, collecting data and characterizing the environment in that location.  In this case the rover is not limited in its sampling capacity.  However the decision to sample is based on a pre-programmed threshold.  While this may be an excellent way to encode subject matter experts' beliefs on what is interesting it is fragile in the face of a changing environment, and does not adapt to the actual environment the rover encounters.
 
% 
% 	- Previous work in robotic exploration is either dependent on global information or it does not make reasoned decisions about the rest of the world and the sampling opportunities that are immediately available to it.
% 			- opportunistic science only takes advantage of what is immediately availble to the robot and with surplus sampling budget.
% 			- whoi and mbari follow set patterns (which is fine, the lawnmower is a good and noble thing) but they use an arbitrary threshold to make the decision to sample or not.  They do not take into account the rest of the environment.
% 

In \cite{gridhar2013autonomous} and \cite{girdhar2013Aautonomous} Girdhar
\emph{et al.} present an approach to autonomous exploration where the robot
investigates a scene when it encounters phenomena that do not reflect its
current model of the world.  Specifically they use topic models to describe
scenes, and when they encounter scenes that do not fit into the topic models
they have constructed.  In these works the vehicle has no limit on its sampling
capacity and is always collecting data.  By slowing the vehicle down more
samples are collected in anomalous scenes.  In this fashion this is very
similar to the work in \cite{thompson2013adaptive}.

Additionally Girdhar \emph{et al.} use their anomaly detection techniques in
\cite{girdhar2013autonomous} to develop a path planning method to maximize
information gain of paths \cite{girdhar2014curiosity}.  In that respect it is
belongs with the family of curiosity-driven algorithms pioneered by
Schmidh{\"u}ber \emph{et al.} \cite{schmidhuber1997what},
\cite{schmidhuber2003exploring}, \cite{schmidhuber2009simple},
\cite{sun2011planning}.  The fundamental concept behind these approaches is
that an explorer should spend their time investigating regions of the world (or
hypothesis space) where their models are the least certain.

\subsubsection{Our Prior Work}
		- Apply optimal foraging, specifically Charnov's marginal value theorem
		model of foraging to the problem of science autonomy* in the following setting:	
			- Limited sampling budget
			- Attempting to learn distributions not acquire quantities of some object.
			- Trying to learn information objectively without an objective. 
			- Can't know the reward until you sample.
			- It's a fusion of bandits and foraging.  The fact that we are focusing on gaining information gives us diminishing returns which lets us use 
			- Don't have access to all the arms at any given time, which is what makes it a foraging problem.
			- Don't have the choice of what objects you encounter.

As seen above real robots may not be able to predict the rewards they will earn from their actions and have to deal with unreliable arrival rates for sampling opportunities.  These are concerns that are not modelled in typical sequential experiment selection algorithms such as the Multi-armed bandit or secretary problems.  This motivates the problem setting used in this paper, and that describe in detail in the following section.
