%% main text
\section{Introduction}
\label{sec:intro}

\begin{enumerate}
\item Why did I do the work?
	Robots exploring the world right now either depend highly on their controllers to give them objectives or they are planning with some global knowledge.  Operating in these kinds of conditions puts constraints on robot exploration opterations by relying on either humans to make decisions or a significant amount of scouting.

	Relying on humans to make decisions means that remote operators require considerable bandwidth to acquire sufficient situational awareness.  Conducting sufficient reconnaissance to make good decisions often obviates the need to send a robotic agent.

	What is lacking in the literature are robots that make decisions about what to investigate \emph{in situ} without reliance on humans and without necessarily having global knowledge.
	\item What were the central motivations and hypotheses?
	Animals, e.g. human geologists, make decisions about investigating phenomena in the world without necessarily having access to high resolution satellite imagery.  Despite this lack they are able to chose between sampling from materials in front of them and moving on to determine more profitable sampling locations.

	While these decisions may not be globally optimal they do demonstrate an ability that is lacking from exploration robots: to make decisions to stop and engage with the environment or to continue travelling in the hopes of finding more informative sampling locations.
	\end{enumerate}



